%%
% Plantilla de Memoria
% Modificación de una plantilla de Latex de Nicolas Diaz para adaptarla 
% al castellano y a las necesidades de escribir informática y matemáticas.
%
% Editada por: Mario Román
%
% License:
% CC BY-NC-SA 3.0 (http://creativecommons.org/licenses/by-nc-sa/3.0/)
%%

%%%%%%%%%%%%%%%%%%%%%
% Thin Sectioned Essay
% LaTeX Template
% Version 1.0 (3/8/13)
%
% This template has been downloaded from:
% http://www.LaTeXTemplates.com
%
% Original Author:
% Nicolas Diaz (nsdiaz@uc.cl) with extensive modifications by:
% Vel (vel@latextemplates.com)
%
% License:
% CC BY-NC-SA 3.0 (http://creativecommons.org/licenses/by-nc-sa/3.0/)
%
%%%%%%%%%%%%%%%%%%%%%

%----------------------------------------------------------------------------------------
%	PAQUETES Y CONFIGURACIÓN DEL DOCUMENTO
%----------------------------------------------------------------------------------------

%% Configuración del papel.
% microtype: Tipografía.
% mathpazo: Usa la fuente Palatino.
\documentclass[a4paper, 10pt]{article}
\usepackage[protrusion=true,expansion=true]{microtype}
\usepackage{mathpazo}

% Requisitos de formato pedidos por el profesor:
\renewcommand{\rmdefault}{phv} % Arial
\renewcommand{\sfdefault}{phv} % Arial
\usepackage[top=2.5cm, bottom=2.5cm, left=3cm, right=3cm]{geometry} % Márgenes

% Hipervínculos
\usepackage[hidelinks]{hyperref}

% Indentación de párrafos para Palatino
\setlength{\parindent}{0pt}
  \parskip=8pt
\linespread{1.05} % Change line spacing here, Palatino benefits from a slight increase by default


%% Castellano.
% noquoting: Permite uso de comillas no españolas.
% lcroman: Permite la enumeración con numerales romanos en minúscula.
% fontenc: Usa la fuente completa para que pueda copiarse correctamente del pdf.
\usepackage[spanish,es-noquoting,es-lcroman]{babel}
\usepackage[utf8]{inputenc}
\usepackage[T1]{fontenc}
\selectlanguage{spanish}


%% Gráficos
\usepackage{graphicx} % Required for including pictures
\usepackage{wrapfig} % Allows in-line images
\usepackage[usenames,dvipsnames]{color} % Coloring code


%% Matemáticas
\usepackage{amsmath}


%% Bibliografía
\makeatletter
\renewcommand\@biblabel[1]{\textbf{#1.}} % Change the square brackets for each bibliography item from '[1]' to '1.'
\renewcommand{\@listI}{\itemsep=0pt} % Reduce the space between items in the itemize and enumerate environments and the bibliography



%----------------------------------------------------------------------------------------
%	TÍTULO
%----------------------------------------------------------------------------------------
% Configuraciones para el título.
% El título no debe editarse aquí.
\renewcommand{\maketitle}{
  \begin{flushright} % Right align
  
  {\LARGE\@title} % Increase the font size of the title
  
  \vspace{50pt} % Some vertical space between the title and author name
  
  {\large\@author} % Author name
  \\\@date % Date
  \vspace{40pt} % Some vertical space between the author block and abstract
  \end{flushright}
}

% Título
\title{\textbf{Ingeniería de Servidores}\\ % Title
Análisis comparativo de Tomcat, JBoss y Glassfish} % Subtitle

		%Comentado ya que se supone que es anónimo
\author{ %\textsc{Óscar Bermúdez Garrido} % Author
%\\
{\textit{Universidad de Granada}}} % Institution

\date{\today} % Date



%----------------------------------------------------------------------------------------
%	DOCUMENTO
%----------------------------------------------------------------------------------------

\begin{document}

\maketitle % Print the title section

% Resumen (Descomentar para usarlo)
\renewcommand{\abstractname}{Resumen} % Uncomment to change the name of the abstract to something else
\begin{abstract}
\end{abstract}

% Palabras clave
%\hspace*{3,6mm}\textit{Keywords:} lorem , ipsum , dolor , sit amet , lectus % Keywords
%\vspace{30pt} % Some vertical space between the abstract and first section


% Índice (Descomentar para usarlo)
%{\parskip=2pt
%  \tableofcontents
%}
%\pagebreak

%% Inicio del documento

\section{Introducción}
	La sociedad de nuestros días se caracteriza por la obtención, almacenamiento y
	utilización de la información. Por tanto, los sistemas encargados de la utilización
	de dicha información causan gran impacto sobre ella. Estos sistemas se desarrollan
	de forma muy rápida y, al igual que ellos, sus necesidades de capacidades hardware
	para su correcto funcionamiento.
	
	Esto se traduce en una necesidad de desarrollo forzado de las plataformas que dan
	soporte a dichas tecnologías. Y este desarrollo, a su vez, ocasiona problemas que,
	al solventarlos, dan lugar a nuevas tecnologías. Uno de estos problemas sería el de
	la comunicación entre sistemas diferentes, ya sea por su sistema operativo, sus
	capacidades hardware o su adaptabilidad y tolerancia a errores.
	
	Una de las posibles soluciones a dicho problema(pero no la única) es la utilización
	de servicios webs. Quizás la mayor de la utilización de servidores web como solución
	a dicho problema es la posibilidad de interacción entre aplicaciones implementadas
	entre diferentes sistemas operativos(o el mismo), esto es, la interoperatibilidad que
	es aprovechada por los sistemas distribuidos. Para favorecer dicha interoperabilidad,
	se fomenta el desarrollo de estándares y protocolos que definan cómo deben comunicarse
	los procesos para obtener una colaboración satisfactoria y simplificada, permitiendo
	así maximizar la eficiencia del servicio.
	
	En el desarrollo de servidores webs, se requieren una serie de instrucciones de
	administración de servidores como puede ser control de seguridad, manejo de
	transacciones, escabilidad, concurrencia, gestión de complementos desplegados,
	gestión de peticiones de clientes,\dots que no siempre son fáciles de gestionar por
	el desarrollador. Aunque el desarrollador sea capaz de administrar adecuadamente el
	servidor web, éste esfuerzo adicional distrae la atención sobre la lógica de negocio,
	que debería ser el centro de desarrollo de nuestro servidor web.
	
	Para facilitar el desarrollo de los servidores webs, se desarrollaron plataformas y
	protocolos que se encargan de dichas labores de mantenimiento de bajo nivel permitiendo
	centrar las preocupaciones del desarrollador en la lógica de negocio. Ejemplos de estas
	plataformas serían las plataformas J2EE, Appaserver, .NET, GNUstepWeb, Happstack y
	Django-cms. Todas estas alternativas se basan en el desarrollo de clientes y servicios
	webs. En este documento, nos centraremos en la plataforma J2EE aunque también se podría
	exponer gran variedad de servidores de aplicaciones basados en cualquiera de las otras
	plataformas.
	
	Dentro de la plataforma J2EE, podemos encontrar un gran abanico de posibilidades para
	escoger. Entre ellas, destacan Apache Geronimo, Apache TomEE, Glassfish, WildFly
	(anteriormente conocido como JBoss), WebSphere AS, WebLogic Server y JOnAS application
	server. Esto ocasiona grandes dudas sobre cuál se habitúa mejor a nuestras necesidades,
	que se ven ampliadas por la mitificación de las propiedades que dichas aplicaciones
	poseen.
	
	En este proyecto, se tratarán de conocer los puntos fuertes y puntos débiles de algunas
	de las plataformas anteriormente citadas. En concreto, se hará sobre Apache TomEE(la
	extensión más inmediata del contenedor de servlets Tomcat con los requerimientos de
	J2EE), Glassfish y WildFly, todas ellas son proyectos de software libre implementados
	en Java, siguen los estándares definidos por la plataforma J2EE y son multiplataforma.
	
	
\section{Definiciones previas}
	Antes de comenzar, conviene dejar claros algunos conceptos importantes que irán
	apareciendo	a lo largo del documento\footnote{Aunque no lo indique explícitamente,
	estas definiciones son copia, traducción y/o resumen de la página que doy como
	referencia.}:

	\begin{itemize}
		\item \textbf{J2EE}(también conocido como \textit{Java Platform, Enterprise
		Edition} o \textit{Java EE}): es una plataforma de programación de desarrollo y
		ejecución de aplicaciones Java. Para facilitar dicho desarrollo, dispone de una
		serie de API's(RMI, JMS,...) así como servlets y JavaService Pages que se encargan
		de las tareas de mantenimiento de bajo nivel(seguridad, escalabilidad,
		concurrencia,...) permitiendo al desarrollar centrarse en la lógica de negocio.\cite{J2EE_Def}
		
		\item \textbf{API}(\textit{Application Programming Interface}): Conjunto de rutinas,
		protocolos y herramientas para facilitar el desarrollo de aplicaciones software.
		Ejemplos de API's serían los estándar POSIX o la STL de C++.\cite{API_Def}
		
		\item \textbf{RMI}(\textit{Remote Method Invocation}): API de Java que permite la
		llamada a métodos de forma remota, lo que permite la comunicación entre servidores
		basados en Java.\cite{RMI_Def}
		
		\item \textbf{JMS}(\textit{Java Message Service}): API de Java que permite el envío
		y la recepción de mensajes entre clientes.\cite{JMS_Def} Puede ser:
		\begin{itemize}
			\item Punto a Punto: un cliente le envía un mensaje a otro.
			\item Publicar/Suscribir: un cliente envía el mensaje y varios lo reciben.
		\end{itemize}

		\item \textbf{Servlet}: Programa implementado en Java que se ejecuta en un servidor
		(usualmente en un servidor web) y que responde a las peticiones de los clientes,
		extendiendo así las capacidades del servidor.\cite{Servlet_Def}
		
		\item \textbf{JSP}(\textit{JavaServer Pages}): Tecnología que crea páginas web
		generadas dinámicamente.\cite{JSP_Def}
		
		Para ejecutar JSP en un servidor web compatible se requiere un contenedor de
		servlet(como Tomcat, del que hablaremos más adelante).
		
		\item \textbf{Lógica de negocio}(\textit{Business Logic}): Parte del programa
		que se encarga de codificar las reglas del negocio del mundo real sobre cómo
		se crea, muestra, almacena y cambia la información.\cite{BL_Def}
		
	\end{itemize}

\section{Tomcat}

	\subsection{Apache TomEE}

\section{WildFly}


\section{Glassfish}

\section{Perspectivas futuras}

\newpage
\begin{thebibliography}{10}
	\expandafter\ifx\csname url\endcsname\relax
	  \def\url#1{\texttt{#1}}\fi
	\expandafter\ifx\csname urlprefix\endcsname\relax\def\urlprefix{URL }\fi
	\expandafter\ifx\csname href\endcsname\relax
	  \def\href#1#2{#2} \def\path#1{#1}\fi
	
	\bibitem{J2EE_Def}
	Java Platform, Enterprise Edition. Wikipedia, the free encyclopedia.\\
		\url{https://en.wikipedia.org/wiki/Java_Platform,_Enterprise_Edition}
	
	\bibitem{API_Def}
	Application Programming Interface. Wikipedia, the free encyclopedia.\\
		\url{https://en.wikipedia.org/wiki/Application_programming_interface}
	
	\bibitem{RMI_Def}
	Java Remote Method Invocation. Wikipedia, the free encyclopedia.\\
		\url{https://en.wikipedia.org/wiki/Java_remote_method_invocation}
	
	\bibitem{JMS_Def}
	Java Message Service. Wikipedia, the free encyclopedia.\\
		\url{https://en.wikipedia.org/wiki/Java_Message_Service}
	
	\bibitem{Servlet_Def}
	ORACLE Help Center.\\
		\url{http://docs.oracle.com/javaee/6/api/javax/servlet/Servlet.html}
	
	\bibitem{JSP_Def}
	JavaServer Pages. Wikipedia, the free encyclopedia.\\
		\url{https://en.wikipedia.org/wiki/JavaServer_Pages}
	
	\bibitem{BL_Def}
	Business Logic. Wikipedia, the free encyclopedia.\\
		\url{https://en.wikipedia.org/wiki/Business_logic}
	
\end{thebibliography}

\end{document}
